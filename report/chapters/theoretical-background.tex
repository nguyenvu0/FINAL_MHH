\section{Theoretical Background}
\label{sec:theoretical}

\subsection{Petri Nets}
\label{subsec:petri-nets}

\subsubsection{Basic Definitions}
A \textbf{Petri net} is a tuple $N = (P, T, F, W, M_0)$ where:
\begin{itemize}
    \item $P$ is a finite set of \textbf{places} (conditions)
    \item $T$ is a finite set of \textbf{transitions} (events), where $P \cap T = \emptyset$
    \item $F \subseteq (P \times T) \cup (T \times P)$ is the \textbf{flow relation} (arcs)
    \item $W: F \to \mathbb{N}^+$ is the \textbf{weight function}
    \item $M_0: P \to \mathbb{N}$ is the \textbf{initial marking}
\end{itemize}

For a transition $t \in T$:
\begin{itemize}
    \item The \textbf{preset} is ${}^{\bullet}t = \{p \in P \mid (p,t) \in F\}$
    \item The \textbf{postset} is $t^{\bullet} = \{p \in P \mid (t,p) \in F\}$
\end{itemize}

\subsubsection{Firing Semantics}
A marking $M: P \to \mathbb{N}$ assigns a non-negative integer (number of tokens) to each place.

A transition $t$ is \textbf{enabled} at marking $M$ if:
\begin{equation}
    \forall p \in {}^{\bullet}t: M(p) \geq W(p,t)
\end{equation}

When an enabled transition $t$ fires, producing a new marking $M'$:
\begin{equation}
    M'(p) = M(p) - W(p,t) + W(t,p)
\end{equation}

\subsubsection{1-Safe Petri Nets}
A Petri net is \textbf{1-safe} (or 1-bounded) if for all reachable markings $M$ and all places $p$:
\begin{equation}
    M(p) \leq 1
\end{equation}

This assignment focuses exclusively on 1-safe nets, which can be efficiently represented using Boolean variables.

\subsubsection{Reachability}
The \textbf{reachability set} $\mathcal{R}(M_0)$ from initial marking $M_0$ is:
\begin{equation}
    \mathcal{R}(M_0) = \{M \mid M_0 \xrightarrow{*} M\}
\end{equation}
where $\xrightarrow{*}$ denotes the reflexive and transitive closure of the firing relation.

\subsection{Binary Decision Diagrams (BDDs)}
\label{subsec:bdds}

\subsubsection{Motivation}
Explicit enumeration of reachable markings suffers from state-space explosion. For a 1-safe net with $|P|$ places, the worst-case number of markings is $2^{|P|}$.

Binary Decision Diagrams (BDDs)~\cite{bryant1986} provide a \textbf{canonical} and often \textbf{compact} representation of Boolean functions. They exploit structural similarities to represent exponentially large sets symbolically.

\subsubsection{Definition}
A BDD is a directed acyclic graph (DAG) with:
\begin{itemize}
    \item Two terminal nodes: $\mathbf{0}$ (false) and $\mathbf{1}$ (true)
    \item Internal nodes labeled with Boolean variables $x_i$
    \item Each internal node has two outgoing edges: low (dashed, $x_i = 0$) and high (solid, $x_i = 1$)
\end{itemize}

\textbf{Reduced Ordered BDDs (ROBDDs)} enforce:
\begin{enumerate}
    \item Fixed variable ordering along all paths
    \item No duplicate nodes (isomorphic subgraphs are merged)
    \item No redundant tests ($x_i$ node where both children are identical)
\end{enumerate}

\subsubsection{Symbolic Reachability Using BDDs}
For a 1-safe Petri net:
\begin{itemize}
    \item Each place $p_i$ is encoded as a Boolean variable $x_i$
    \item A marking $M$ is represented as a Boolean vector $(x_1, \ldots, x_{|P|})$
    \item The reachability set $\mathcal{R}(M_0)$ is encoded as a BDD $R(x)$
\end{itemize}

\textbf{Transition Relation}: For each transition $t$, we build:
\begin{equation}
    T_t(x, x') = \text{Enabled}_t(x) \land \text{NextState}_t(x, x')
\end{equation}
where $x'$ denotes next-state variables.

The global transition relation is:
\begin{equation}
    T(x, x') = \bigvee_{t \in T} T_t(x, x')
\end{equation}

\textbf{Fixed-Point Iteration}:
\begin{algorithmic}[1]
\State $R \gets M_0$ \Comment{Initial BDD}
\Repeat
    \State $R_{\text{new}} \gets R \lor \exists x. (R(x) \land T(x, x'))[x'/x]$
\Until{$R_{\text{new}} \equiv R$}
\State \Return $R$
\end{algorithmic}

\subsection{Integer Linear Programming (ILP)}
\label{subsec:ilp}

\subsubsection{General Form}
An Integer Linear Program aims to:
\begin{align}
    \text{maximize (or minimize)} \quad & c^T x \\
    \text{subject to} \quad & Ax \leq b \\
    & x \in \mathbb{Z}^n
\end{align}
where $c \in \mathbb{Z}^n$, $A \in \mathbb{Z}^{m \times n}$, and $b \in \mathbb{Z}^m$.

\subsubsection{Application to Petri Nets}
ILP can encode Petri net properties using:
\begin{itemize}
    \item \textbf{State variables}: Binary indicators for markings or places
    \item \textbf{Transition variables}: Binary indicators for firing sequences
    \item \textbf{Structural constraints}: Incidence matrix equations
    \item \textbf{Objective functions}: Optimize token counts, minimize deadlock distance, etc.
\end{itemize}

For 1-safe nets, variables are naturally binary: $x_i \in \{0, 1\}$.

\subsection{Combining BDD and ILP}
\label{subsec:bdd-ilp-combination}

The key insight is to leverage the strengths of both approaches:
\begin{enumerate}
    \item \textbf{BDD for enumeration}: Efficiently represent and enumerate the reachability set $\mathcal{R}(M_0)$ symbolically.
    \item \textbf{ILP for optimization/verification}: Formulate precise constraints and objectives over the candidate markings extracted from the BDD.
\end{enumerate}

\textbf{Workflow}:
\begin{enumerate}
    \item Use BDD to construct the symbolic reachability set.
    \item Sample or enumerate candidate markings from the BDD (up to a manageable limit).
    \item Formulate an ILP problem using these candidates as constraints or objectives.
    \item Solve the ILP to find deadlocks, optimal markings, or verify properties.
\end{enumerate}

This hybrid approach balances the compactness of BDDs with the expressiveness of ILP.
